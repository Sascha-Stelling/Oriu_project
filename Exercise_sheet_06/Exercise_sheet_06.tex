\documentclass[10pt,a4]{article}

\usepackage[utf8]{inputenc}
\usepackage{amsmath}
\usepackage{amsfonts}
\usepackage{amssymb}
\usepackage{graphicx}
\usepackage{enumitem}
\usepackage{listings}
\usepackage{hyperref}

\author{Shivali Dubey, Sascha Stelling}
\title{Object Recognition and Image Understanding Exercise Sheet 6}

\begin{document}

\maketitle

\paragraph{Question 1}
\begin{itemize}
	\item \textbf{Team:} \\Shivali:
	\begin{itemize}
		\item Implementation
		\item Neural network setup
	\end{itemize}
	Sascha:
	\begin{itemize}
		\item Implementation
		\item Preparation of the dataset
		\item Running training phase
	\end{itemize}
	\item \textbf{Problem Definition:} \textit{Object detection and multiclass image classification.}
	\\Label images from a set of labels with the assumption that that each images can only get labelled by one class
	\item \textbf{Dataset:} Tiny ImageNet
	\item \textbf{Approach:} Extract features from an input image into a learned filterbank using CNNs containing convolution layers,  PReLU\footnote{\url{https://www.cv-foundation.org/openaccess/content_iccv_2015/papers/He_Delving_Deep_into_ICCV_2015_paper.pdf?spm=5176.100239.blogcont55892.28.pm8zm1&file=He_Delving_Deep_into_ICCV_2015_paper.pdf}} as an activation function and pooling layers and pooling layers using max pooling to reduce the output size for the next neuron and because the exact location of a feature is less important than the rough location relative to other features. 
	For the first layer extract edges using a Harris detector.
	\item \textbf{Evaluation $\mathbf{\&}$ Expected  Results:} Calculate training error and minimize it while maximizing accuracy. We hope to get an accuracy of at least 0.7
	\item \textbf{Hardware:}
	\item \textbf{Excluded Presentation Date:}	
\end{itemize}

\end{document}